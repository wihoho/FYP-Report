\section {Video recognition}
As introduced in the section Image Recognition, recognition tasks could be simplified into a problem to calculate distances between different items in the evaluated data set. This is because once the distance matrix is calculated, this distance matrix could be input into K-nearest neighbor algorithms or better approaches like Support Vector Machine for recognition. However, it is not so easy to compare two raw videos in quantitative way and thus difficult to calculate distances using raw video data. In order to resolve this problem, a compact representation of each video clip is needed. 
\subsection{Representations of videos}
	\subsubsection{Bag of words}
	\subsubsection{Gaussian mixture models}
\subsection{Distance calculation}
	\subsubsection{Aligned space-time pyramid matching}
	\subsubsection{Distances between Gaussian mixture models}
\subsection{Kernels for classification}
\subsection{Other approach: concept attributes}


